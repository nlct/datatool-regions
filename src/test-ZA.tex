% arara: pdflatex
%\listfiles
\documentclass{article}

\usepackage[T1]{fontenc}
\usepackage[afrikaans,english]{babel}
\usepackage[locales={ZA}]{datatool-base}

\begin{document}
Dialects:
\ForEachTrackedDialect\thisdialect
{
\par
\thisdialect:
language tag: \GetTrackedLanguageTag{\thisdialect};
language: \TrackedLanguageFromDialect{\thisdialect};
region: \TrackedIsoCodeFromLanguage{\TwoLetterIsoCountryCode}{\thisdialect}.
 }

\selectlanguage{afrikaans}
Currency: \DTLdecimaltocurrency{1234.56}{\result}\result.

\DTLparse\result{R12}\result.
Data type: \DTLgetDataTypeName{\DTLdatumtype{\result}}.
Numeric value: \DTLdatumvalue{\result}.

\DTLsetup{datetime={parse}}
\DTLsetLocaleOptions{ZA}{
 date-style=dmyyyy,
 date-variant = slash,
 time-variant = colon
}
\DTLparse\result{1/2/2025 11:45 SAST}
String: \result.
Data type: \DTLgetDataTypeName{\DTLdatumtype{\result}}.
Numeric value: \DTLdatumvalue{\result}.


\selectlanguage{english}
Currency: \DTLdecimaltocurrency{1234.56}{\result}\result.

\DTLparse\result{R12}\result.
Data type: \DTLgetDataTypeName{\DTLdatumtype{\result}}.
Numeric value: \DTLdatumvalue{\result}.

\DTLsetup{datetime={parse}}
\DTLsetLocaleOptions{ZA}{
 date-style=dmyyyy,
 date-variant = dialect,
 time-variant = colon
}
\DTLparse\result{1 June 2025}
String: \result.
Data type: \DTLgetDataTypeName{\DTLdatumtype{\result}}.
Numeric value: \DTLdatumvalue{\result}.

\end{document}
