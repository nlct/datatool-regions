% arara: pdflatex
\PassOptionsToPackage{noprint}{doc}% command names are too long
\documentclass{ltxdoc}

\usepackage[T1]{fontenc}
\usepackage{tcolorbox}
\usepackage[left=2.5cm,right=2.5cm]{geometry}

\CheckSum{0}

\newcommand*{\sty}[1]{\textsf{#1}}
\newcommand*{\file}[1]{\texorpdfstring{\nolinkurl{#1}}{#1}}
\newcommand*{\filemeta}[3]{%
 \texorpdfstring
  {\nolinkurl{#1}\meta{#2}\nolinkurl{#3}}%
  {#1<#2>#3}%
}
\newcommand*{\filemetameta}[5]{%
 \texorpdfstring
 {\nolinkurl{#1}\meta{#2}\nolinkurl{#3}\meta{#4}\nolinkurl{#5}}%
 {#1<#2>#3<#4>#5}%
}
\newcommand*{\opt}[1]{\textsf{#1}}
\newcommand*{\qt}[1]{“#1”}
\newcommand*{\setlocaleopts}[1]{%
 \cs{DTLsetLocaleOptions}\allowbreak
 \texttt{\brackets{\strong{#1}}}\allowbreak
 \marg{key=val list}%
}

\definecolor{defbackground}{rgb}{1,1,0.75}
\newtcolorbox{definition}{colback=defbackground}
\newtcolorbox{important}{colback=red!5!white,colframe=red}

\tcbuselibrary{documentation}
\tcbset{verbatim ignore percent}

\RecordChanges
\PageIndex

\title{Regions Support for \sty{datatool} Package}
\author{Nicola L. C. Talbot}
%%DATECMD%%

\begin{document}
\maketitle

\begin{abstract}
This is the regions support for the \sty{datatool}
package (version 3.0+). This needs to be installed in addition to
\sty{datatool}. To ensure language support, you will also need to
install the applicable language module (for example,
\sty{datatool-english}).
\end{abstract}

\tableofcontents

\section{Introduction}
\label{sec:intro}

This bundle provides language-independent region support files for
\sty{datatool} v3.0+. The files simply need to be installed on \TeX's path.
(They will be ignored if a pre-3.0 version of \sty{datatool} is installed.)
The \sty{datatool-base} package (which is automatically loaded by
\sty{datatool}) uses \sty{tracklang}'s interface for detecting
localisation settings and finding the appropriate files.
If you use \sty{babel} or \sty{polyglossia}, make sure that you
specify the document dialects before the first package to load
\sty{tracklang}.

\begin{important}
If the chosen document language does not have an associated region,
no region support will be provided.
\end{important}

For example:
\begin{dispListing}
\usepackage[british]{babel}
\usepackage{datatool-base}
\end{dispListing}
In this case, \opt{british} is associated with region \qt{GB} so
\sty{datatool-base} will load \file{datatool-GB.ldf} if it's on
\TeX's path.

Alternatively, if you are not using a language package, simply use
the \opt{locales} option. For example:
\begin{dispListing}
\usepackage[locales={en-GB}]{datatool-base}
\end{dispListing}

The \sty{tracklang} interface doesn't allow the language to be
omitted, but \sty{datatool-base}'s \opt{locales} option will prefix
a languageless region with \opt{und} (undetermined) before passing
it to \sty{tracklang}. For example:
\begin{dispListing}
\usepackage[locales={GB}]{datatool-base}
\end{dispListing}
This is equivalent to:
\begin{dispListing}
\usepackage[locales={und-GB}]{datatool-base}
\end{dispListing}

In general, it's best to include the language to ensure the
appropriate language files are also loaded, if they are available.

Any option that can be passed to \sty{datatool-base} can also be
passed to \sty{datatool} but if \sty{datatool-base} has already been
loaded, it will be too late to use the \opt{locales} option.
For example:
\begin{dispListing}
\usepackage[locales={en-GB}]{datatool}
\end{dispListing}
But not:
\begin{dispListing*}{title={Incorrect!},colframe=red}
\usepackage{datatool-base}
\usepackage[locales={en-GB}]{datatool}
\end{dispListing*}

If another package that also loads \sty{tracklang} is loaded first,
then \sty{datatool-base} can pick up the settings from that. For
example:
\begin{dispListing}
\usepackage[en-GB]{datetime2}
\usepackage{datatool}
\end{dispListing}

Supplementary packages provided with \sty{datatool} can also have
the locales provided. For example:
\begin{dispListing}
\usepackage[locales={en-GB}]{datagidx}
\end{dispListing}
As with \sty{datatool}, these supplementary packages internally load
\sty{datatool-base} so if that has already been loaded, then the
localisation support should already have been set.

\section{Region \qt{CA}}

The \file{datatool-CA.ldf} file provides support for region \qt{CA}
(Canada). This supplies the currency (CAD) but number formatting
depends on the language, so this requires specific language \&
region files, which should be provided by the applicable language
module. For example, \sty{datatool-english} provides
\file{datatool-en-CA.ldf}.

\section{Region \qt{GB}}

The \file{datatool-GB.ldf} file provides support for region \qt{GB}
(United Kingdom).

\section{Region \qt{ZA}}

The \file{datatool-ZA.ldf} file provides support for region \qt{ZA}
(South Africa). This supplies the currency (ZAR) but number formatting
depends on the language, so this requires specific language \&
region files, which should be provided by the applicable language
module. For example, \sty{datatool-english} provides
\file{datatool-en-ZA.ldf}.

\StopEventually{%
  \PrintChanges
  \PrintIndex
}
\end{document}
