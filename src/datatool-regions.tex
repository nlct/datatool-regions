% arara: lualatex
% arara: makeindex
% arara: lualatex
\listfiles
\documentclass[titlepage=false]{scrreport}
\usepackage{doc}

\usepackage{fontspec}
\usepackage{tcolorbox}
\usepackage[left=2.5cm,right=2.5cm]{geometry}

\usepackage
 [
   locales=
   {
     en-AU,
     en-CA,
     en-FK,
     en-GB,
     en-IE,
     en-US,
     en-ZA
   },
   verbose
 ]
 {datatool-base}

\setromanfont{Noto Serif}
\setsansfont{Noto Sans}
\setmonofont{Noto Sans Mono}

\CheckSum{0}

\newcommand*{\sty}[1]{\textsf{#1}}
\newcommand*{\file}[1]{\texorpdfstring{\nolinkurl{#1}}{#1}}
\newcommand*{\ldf}[1]{\file{datatool-#1.ldf}}
\newcommand*{\filemeta}[3]{%
 \texorpdfstring
  {\nolinkurl{#1}\meta{#2}\nolinkurl{#3}}%
  {#1<#2>#3}%
}
\newcommand*{\filemetameta}[5]{%
 \texorpdfstring
 {\nolinkurl{#1}\meta{#2}\nolinkurl{#3}\meta{#4}\nolinkurl{#5}}%
 {#1<#2>#3<#4>#5}%
}
\newcommand*{\opt}[1]{\textsf{#1}}
\newcommand*{\qt}[1]{“#1”}
\newcommand*{\setbaseopt}[1]{%
 \cs{DTLsetup}\allowbreak
 \texttt{\brackets{\allowbreak#1}}%
}
\newcommand*{\setlocaleopts}[1]{%
 \cs{DTLsetLocaleOptions}\allowbreak
 \texttt{\brackets{\strong{#1}}}\allowbreak
 \marg{key=val list}%
}

\newcommand*{\setlocaleopt}[2]{%
 \cs{DTLsetLocaleOptions}\allowbreak
 \texttt{\brackets{#1}\allowbreak\brackets{\allowbreak#2}}%
}

\definecolor{defbackground}{rgb}{1,1,0.75}
\newtcolorbox{definition}{colback=defbackground}
\newtcolorbox{important}{colback=red!5!white,colframe=red}

\tcbuselibrary{documentation}
\tcbset{verbatim ignore percent}

\RecordChanges
\PageIndex

\appto\MacroFont{\footnotesize}

\title{Regions Support for \sty{datatool} Package}
\author{Nicola L. C. Talbot}
%%DATECMD%%

\begin{document}
\maketitle

\begin{abstract}
This is the regions support for the \sty{datatool}
package (version 3.0+). This needs to be installed in addition to
\sty{datatool}. To ensure language support, you will also need to
install the applicable language module (for example,
\sty{datatool-english}).
\end{abstract}

\tableofcontents

\chapter{Introduction}
\label{sec:intro}

This bundle provides language-independent region support files for
\sty{datatool} v3.0+. The files simply need to be installed on \TeX's path.
(They will be ignored if a pre-3.0 version of \sty{datatool} is installed.)
The \sty{datatool-base} package (which is automatically loaded by
\sty{datatool}) uses \sty{tracklang}'s interface for detecting
localisation settings and finding the appropriate files.
If you use \sty{babel} or \sty{polyglossia}, make sure that you
specify the document dialects before the first package to load
\sty{tracklang}.

\begin{important}
If the chosen document language does not have an associated region,
no region support will be provided.
\end{important}

For example:
\begin{dispListing}
\usepackage[british]{babel}
\usepackage{datatool-base}
\end{dispListing}
In this case, \opt{british} is associated with region \qt{GB} so
\sty{datatool-base} will load \file{datatool-GB.ldf} if it's on
\TeX's path.

Alternatively, if you are not using a language package, simply use
the \opt{locales} option. For example:
\begin{dispListing}
\usepackage[locales={en-GB}]{datatool-base}
\end{dispListing}

The \sty{tracklang} interface doesn't allow the language to be
omitted, but \sty{datatool-base}'s \opt{locales} option will
check for any item in the list that simply consists of two uppercase
letters. If found, the region will be added to any tracked dialects
that don't have a region set. If no dialects have been tracked,
the region will be tracked with the language set to
\opt{und} (undetermined). For example:
\begin{dispListing}
\documentclass{article}
\usepackage[locales={GB}]{datatool-base}
\end{dispListing}
This is equivalent to:
\begin{dispListing}
\usepackage[locales={und-GB}]{datatool-base}
\end{dispListing}
(since no language has been specified). Whereas
\begin{dispListing}
\usepackage[afrikaans,english]{babel}
\usepackage[locales={ZA}]{datatool-base}
\end{dispListing}
will add the region \opt{ZA} to \opt{afrikaans} and \opt{english}
dialects.

If the language already has a region, or if there are multiple
regions for a particular language, then you will need to 
include the language in the tag. For example:
\begin{dispListing}
\usepackage[locales={en-GB,en-IE}]{datatool-base}
\end{dispListing}

\begin{important}
Bear in mind that if \sty{tracklang} can't determine the
applicable dialect label for the captions hook, the settings may not
be applied when the language changes in multilingual documents.
In this case, you can either load \sty{tracklang} before
\sty{datatool-base} and set up the appropriate mappings or 
just add the applicable \texttt{\cs{DTL}\meta{tag}{LocaleHook}} 
command to the relevant captions hook.
\end{important}

Any option that can be passed to \sty{datatool-base} can also be
passed to \sty{datatool} but if \sty{datatool-base} has already been
loaded, it will be too late to use the \opt{locales} option.
For example:
\begin{dispListing}
\usepackage[locales={en-GB}]{datatool}
\end{dispListing}
But not:
\begin{dispListing*}{title={Incorrect!},colframe=red}
\usepackage{datatool-base}
\usepackage[locales={en-GB}]{datatool}
\end{dispListing*}

If another package that also loads \sty{tracklang} is loaded first,
then \sty{datatool-base} can pick up the settings from that. For
example:
\begin{dispListing}
\usepackage[en-GB]{datetime2}
\usepackage{datatool}
\end{dispListing}

Supplementary packages provided with \sty{datatool} can also have
the locales provided. For example:
\begin{dispListing}
\usepackage[locales={en-GB}]{datagidx}
\end{dispListing}
As with \sty{datatool}, these supplementary packages internally load
\sty{datatool-base} so if that has already been loaded, then the
localisation support should already have been set.

\section{Non-Region Specific Settings}

Some settings are not specific to regions, such as
\opt{currency-symbol-style}. These should be set in \cs{DTLsetup}.
All settings relating to numbers and currency are set within the
\opt{numeric} option and all settings relating to dates and times
are set within the \opt{datetime} option. These settings are
described in the \sty{datatool} user guide.

For example:
\begin{dispListing}
\DTLsetup{
 numeric = {
    region-currency-prefix = smallcaps ,
    currency-symbol-style = symbol
  },
 datetime = {
   parse = auto-reformat
 }
}
\end{dispListing}

\chapter{Supported Regions}

Only a limited number of regions are currently supported.

\section{Region \qt{AU}}
\DTLenLocaleHook
\DTLAULocaleHook

The \file{datatool-AU.ldf} file provides support for region \qt{AU}
(Australia). This defines the \qt{AUD} currency and sets it as
the default currency. The default number group character is a
comma, and the default decimal character is a dot.

If there's no caption hook and you have multiple locales, you will
need to set both the language and region hooks. For example, this
document has:
\begin{dispListing}
\DTLenLocaleHook
\DTLAULocaleHook
\end{dispListing}

Options may be set with \setlocaleopts{AU}. Available options are
listed below.

\begin{docKey}[AU]{currency-symbol-prefix}{=\meta{boolean}}{initially \opt{false}}
If true, shows the region prefix before the currency symbol if
the \opt{numeric} setting \opt{currency-symbol-style} is not set to \qt{iso}.
The prefix font may be changed to smallcaps or smaller with
\setbaseopt{ numeric = \brackets{ region-currency-prefix = \meta{value} } }
where \meta{value} is \opt{smallcaps} or \opt{smaller}.
\end{docKey}

\begin{dispExample}
\DTLsetup{numeric={region-currency-prefix=smallcaps}}
\DTLsetLocaleOptions{AU}{currency-symbol-prefix}
\DTLdecimaltocurrency{12345.678}{\result}\result
\end{dispExample}


\begin{docKey}[AU]{currency-symbol-position}{=\meta{value}}{initially \opt{before}}
Adjusts the formatting currency style. If \docValue{before}, the
currency symbol is placed before the value. If \docValue{after}, the
currency symbol is placed after the value.
\end{docKey}

\begin{dispExample}
\DTLsetup{numeric={currency-symbol-style=iso}}
\DTLsetLocaleOptions{AU}{currency-symbol-position=after}
\DTLdecimaltocurrency{12345.678}{\result}\result
\end{dispExample}

\begin{docKey}[AU]{currency-symbol-sep}{=\meta{value}}{initially \opt{none}}
Sets the separator to use between the currency symbol (not code) and the value.
Permitted values: \docValue{none} (no space), \docValue{thin-space}
(a thin space), \docValue{space} (a normal space), or
\docValue{nbsp} (a non-breaking space).
\end{docKey}

\begin{dispExample}
\DTLsetLocaleOptions{AU}{currency-symbol-sep=thin-space}
\DTLdecimaltocurrency{12345.678}{\result}\result
\end{dispExample}

\begin{docKey}[AU]{number-style}{=\meta{setting}}{initially \opt{official}}
Sets the number group and decimal characters. The value may be one
of: \docValue{official} (comma number group and decimal point), or
\docValue{unofficial} (thin space number group and decimal point).
In the case of \docValue{unofficial}, a normal space may also be used
when parsing.
\end{docKey}

\begin{dispExample}
\DTLsetup{numeric={auto-reformat}}
\DTLsetLocaleOptions{AU}{number-style=unofficial}
\DTLdecimaltocurrency{12345.678}{\result}\result.
(Value: \DTLdatumvalue{\result}.)

\DTLparse{\result}{\$12 345.678}\result.
(Value: \DTLdatumvalue{\result}.)
\end{dispExample}

The remaining options relate to dates and times, which are still
experimental. You need to enable date and time parsing with
\setbaseopt{ datetime = \brackets{parse} }.

\begin{docKey}[AU]{date-style}{=\meta{style}}{initially \opt{mdyyyy}}
Sets the current date style. The value may be one of:
\docValue{dmyyyy} (day month year), \docValue{mdyyyy} (month day year),
\docValue{yyyymd} (year month day),
\docValue{dmyy} (day month 2-digit year),
\docValue{mdyy} (month day 2-digit year), or
\docValue{yymd} (2-digit year month day).
\end{docKey}

\begin{docKey}[AU]{date-variant}{=\meta{style}}{initially \opt{slash}}
Sets the current numeric date separator.
Allowed values: \docValue{slash} (\code{/}), \docValue{hyphen}
(\code{-}), \docValue{dot} (\code{.}) or
\docValue{dialect} (if the language supports it).
\end{docKey}

\begin{docKey}[AU]{time-variant}{=\meta{style}}{initially \opt{colon}}
Sets the current numeric time separator.
Allowed values: \docValue{colon} (\code{:}), \docValue{dot} (\code{.}) or
\docValue{dialect} (if the language supports it),
\docValue{dialect-colon} (if the language supports it), or
\docValue{dialect-dot} (if the language supports it).
\end{docKey}

\begin{dispExample}
\DTLsetup{datetime={parse}}
\DTLsetLocaleOptions{AU}{
 date-style=mdyyyy,
 date-variant = dialect,
 time-variant = dot
}
\DTLparse\result{February 28, 2025 3.45pm AEDT}
String: \result.
Data type: \DTLgetDataTypeName{\DTLdatumtype{\result}}.
Numeric value: \DTLdatumvalue{\result}.

\ExplSyntaxOn
\datatool_extract_timestamp:NN \result \l_tmpa_tl
Time-stamp: ~ \l_tmpa_tl
\ExplSyntaxOff
\end{dispExample}

\begin{docCommand}{datatoolAUSetNumberChars}{}
Hook to switch to AU number group and decimal characters.
\end{docCommand}

\begin{docCommand}{datatoolAUcurrencyfmt}{}
Used to format the AUD currency. This supports the currency symbol
prefix.
\end{docCommand}

\begin{docCommand}{DTLAULocaleHook}{}
Hook to switch to AU settings. This may be added to the captions
hook by \sty{datatool-base}, depending on the settings. Otherwise
it can be explicitly used to switch to this region.
\end{docCommand}

\section{Region \qt{CA}}
\DTLenLocaleHook
\DTLenCALocaleHook
\DTLCALocaleHook

The \file{datatool-CA.ldf} file provides support for region \qt{CA}
(Canada). This supplies the currency (CAD) but number formatting
depends on the language, so this requires specific language \&
region files, which should be provided by the applicable language
module. For example, \sty{datatool-english} provides
\file{datatool-en-CA.ldf}.

If there's no caption hook and you have multiple locales, you will
need to set both the language and region hooks. For example, this
document has:
\begin{dispListing}
\DTLenLocaleHook
\DTLenCALocaleHook
\DTLCALocaleHook
\end{dispListing}

Options may be set with \setlocaleopts{CA}. Available options are
listed below.

\begin{docKey}[CA]{currency-symbol-prefix}{=\meta{boolean}}{initially \opt{false}}
If true, shows the region prefix before the currency symbol if
the \opt{numeric} setting \opt{currency-symbol-style} is not set to \qt{iso}.
The prefix font may be changed to smallcaps or smaller with
the base \opt{numeric} option \opt{region-currency-prefix} setting.
\end{docKey}

For example:
\begin{dispExample}
\DTLsetup{numeric={region-currency-prefix=smallcaps}}
\DTLsetLocaleOptions{CA}{currency-symbol-prefix}
\DTLdecimaltocurrency{12345.678}{\result}\result
\end{dispExample}

\begin{docKey}[CA]{currency-symbol-position}{=\meta{value}}{initially \opt{before}}
Adjusts the formatting currency style. If \docValue{before}, the
currency symbol is placed before the value. If \docValue{after}, the
currency symbol is placed after the value.
\end{docKey}

\begin{dispExample}
\DTLsetup{numeric={currency-symbol-style=iso}}
\DTLsetLocaleOptions{CA}{currency-symbol-position=after}
\DTLdecimaltocurrency{12345.678}{\result}\result
\end{dispExample}

\begin{docKey}[CA]{currency-symbol-sep}{=\meta{value}}{initially \opt{none}}
Sets the separator to use between the currency symbol (not code) and the value.
Permitted values: \docValue{none} (no space), \docValue{thin-space}
(a thin space), \docValue{space} (a normal space), or
\docValue{nbsp} (a non-breaking space).
\end{docKey}

\begin{dispExample}
\DTLsetLocaleOptions{CA}{currency-symbol-sep=thin-space}
\DTLdecimaltocurrency{12345.678}{\result}\result
\end{dispExample}

\begin{docKey}[CA]{number-style}{=\meta{style}}{}
This option will attempt to set \opt{number-style}=\meta{style} for
the module \meta{lang}-CA, if it's defined. Where \meta{lang} is the
current language tag. For example, if the current language tag is
\qt{en}, then this will be equivalent to 
\setlocaleopt{en-CA}{number-style=\meta{style}}.
\end{docKey}

For example, if \ldf{en-CA} has been loaded and the current locale
is en-GB:
\begin{dispExample}
\DTLsetLocaleOptions{CA}{number-style=unofficial}
\DTLdecimaltocurrency{12345.678}{\result}\result
\end{dispExample}

The remaining options relate to dates and times, which are still
experimental. You need to enable date and time parsing with
\setbaseopt{ datetime = \brackets{parse} }.

\begin{docKey}[CA]{date-style}{=\meta{style}}{initially \opt{yyyymd}}
Sets the current date style. The value may be one of:
\docValue{dmyyyy} (day month year), \docValue{mdyyyy} (month day year),
\docValue{yyyymd} (year month day),
\docValue{dmyy} (day month 2-digit year),
\docValue{mdyy} (month day 2-digit year), or
\docValue{yymd} (2-digit year month day).
\end{docKey}

\begin{docKey}[CA]{date-variant}{=\meta{style}}{initially \opt{hyphen}}
Sets the current numeric date separator.
Allowed values: \docValue{slash} (\code{/}), \docValue{hyphen}
(\code{-}), \docValue{dot} (\code{.}) or
\docValue{dialect} (if the language supports it).
\end{docKey}

\begin{docKey}[CA]{time-variant}{=\meta{style}}{initially \opt{colon}}
Sets the current numeric time separator.
Allowed values: \docValue{colon} (\code{:}), \docValue{dot} (\code{.}) or
\docValue{dialect} (if the language supports it),
\docValue{dialect-colon} (if the language supports it), or
\docValue{dialect-dot} (if the language supports it).
\end{docKey}

\begin{dispExample}
\DTLsetup{datetime={parse}}
\DTLsetLocaleOptions{CA}{
 date-style=dmyyyy,
 date-variant = dialect,
 time-variant = dot
}
\DTLparse\result{Fri 28th Feb 2025 3.45pm PST}
String: \result.
Data type: \DTLgetDataTypeName{\DTLdatumtype{\result}}.
Numeric value: \DTLdatumvalue{\result}.

\ExplSyntaxOn
\datatool_extract_timestamp:NN \result \l_tmpa_tl
Time-stamp: ~ \l_tmpa_tl
\ExplSyntaxOff
\end{dispExample}

\begin{docCommand}{datatoolCASetNumberChars}{}
Hook to switch to CA number group and decimal characters but
requires language support to be enabled as well otherwise it will
simply trigger a warning.
\end{docCommand}

\begin{docCommand}{datatoolCAcurrencyfmt}{}
Used to format the CAD currency. This supports the currency symbol
prefix.
\end{docCommand}

\begin{docCommand}{DTLCALocaleHook}{}
Hook to switch to CA settings. This may be added to the captions
hook by \sty{datatool-base}, depending on the settings. Otherwise
it can be explicitly used to switch to this region.
\end{docCommand}

\section{Region \qt{FK}}
\DTLenLocaleHook
\DTLFKLocaleHook

The \file{datatool-FK.ldf} file provides support for region \qt{FK}
(Falkland Islands). This defines the \qt{FKP} currency and sets it as
the default currency. The default number group character is a
comma, and the default decimal character is a dot.

If there's no caption hook and you have multiple locales, you will
need to set both the language and region hooks. For example, this
document has:
\begin{dispListing}
\DTLenLocaleHook
\DTLFKLocaleHook
\end{dispListing}

Options may be set with \setlocaleopts{FK}. Available options are
listed below.

\begin{docKey}[FK]{currency-symbol-prefix}{=\meta{boolean}}{initially \opt{false}}
If true, shows the region prefix before the currency symbol if
the \opt{numeric} setting \opt{currency-symbol-style} is not set to \qt{iso}.
The prefix font may be changed to smallcaps or smaller with
\setbaseopt{ numeric = \brackets{ region-currency-prefix = \meta{value} } }
where \meta{value} is \opt{smallcaps} or \opt{smaller}.
\end{docKey}

\begin{dispExample}
\DTLsetup{numeric={region-currency-prefix=smallcaps}}
\DTLsetLocaleOptions{FK}{currency-symbol-prefix}
\DTLdecimaltocurrency{12345.678}{\result}\result
\end{dispExample}


\begin{docKey}[FK]{currency-symbol-position}{=\meta{value}}{initially \opt{before}}
Adjusts the formatting currency style. If \docValue{before}, the
currency symbol is placed before the value. If \docValue{after}, the
currency symbol is placed after the value.
\end{docKey}

\begin{dispExample}
\DTLsetup{numeric={currency-symbol-style=iso}}
\DTLsetLocaleOptions{FK}{currency-symbol-position=after}
\DTLdecimaltocurrency{12345.678}{\result}\result
\end{dispExample}

\begin{docKey}[FK]{currency-symbol-sep}{=\meta{value}}{initially \opt{none}}
Sets the separator to use between the currency symbol (not code) and the value.
Permitted values: \docValue{none} (no space), \docValue{thin-space}
(a thin space), \docValue{space} (a normal space), or
\docValue{nbsp} (a non-breaking space).
\end{docKey}

\begin{dispExample}
\DTLsetLocaleOptions{FK}{currency-symbol-sep=thin-space}
\DTLdecimaltocurrency{12345.678}{\result}\result
\end{dispExample}

\begin{docKey}[FK]{number-style}{=\meta{setting}}{initially \opt{official}}
Sets the number group and decimal characters. The value may be one
of: \docValue{official} (comma number group and decimal point) or
\docValue{unofficial} (thin space number group and decimal point).
In the case of \docValue{unofficial}, a normal space may also be used
when parsing.
\end{docKey}

\begin{dispExample}
\DTLsetup{numeric={auto-reformat}}
\DTLsetLocaleOptions{FK}{number-style=unofficial}
\DTLdecimaltocurrency{12345.678}{\result}\result.
(Value: \DTLdatumvalue{\result}.)

\DTLparse{\result}{£12 345.678}\result.
(Value: \DTLdatumvalue{\result}.)
\end{dispExample}

The remaining options relate to dates and times, which are still
experimental. You need to enable date and time parsing with
\setbaseopt{ datetime = \brackets{parse} }.

\begin{docKey}[FK]{date-style}{=\meta{style}}{initially \opt{dmyyyy}}
Sets the current date style. The value may be one of:
\docValue{dmyyyy} (day month year), \docValue{mdyyyy} (month day year),
\docValue{yyyymd} (year month day),
\docValue{dmyy} (day month 2-digit year),
\docValue{mdyy} (month day 2-digit year), or
\docValue{yymd} (2-digit year month day).
\end{docKey}

\begin{docKey}[FK]{date-variant}{=\meta{style}}{initially \opt{slash}}
Sets the current numeric date separator.
Allowed values: \docValue{slash} (\code{/}), \docValue{hyphen}
(\code{-}), \docValue{dot} (\code{.}) or
\docValue{dialect} (if the language supports it).
\end{docKey}

\begin{docKey}[FK]{time-variant}{=\meta{style}}{initially \opt{colon}}
Sets the current numeric time separator.
Allowed values: \docValue{colon} (\code{:}), \docValue{dot} (\code{.}) or
\docValue{dialect} (if the language supports it),
\docValue{dialect-colon} (if the language supports it), or
\docValue{dialect-dot} (if the language supports it).
\end{docKey}

\begin{dispExample}
\DTLsetup{datetime={parse}}
\DTLsetLocaleOptions{FK}{
 date-style=dmyyyy,
 date-variant = dialect,
 time-variant = dot
}
\DTLparse\result{Fri 28th Feb 2025 3.45pm FKST}
String: \result.
Data type: \DTLgetDataTypeName{\DTLdatumtype{\result}}.
Numeric value: \DTLdatumvalue{\result}.

\ExplSyntaxOn
\datatool_extract_timestamp:NN \result \l_tmpa_tl
Time-stamp: ~ \l_tmpa_tl
\ExplSyntaxOff
\end{dispExample}

\begin{docCommand}{datatoolFKSetNumberChars}{}
Hook to switch to FK number group and decimal characters.
\end{docCommand}

\begin{docCommand}{datatoolFKcurrencyfmt}{}
Used to format the FKP currency. This supports the currency symbol
prefix.
\end{docCommand}

\begin{docCommand}{DTLFKLocaleHook}{}
Hook to switch to FK settings. This may be added to the captions
hook by \sty{datatool-base}, depending on the settings. Otherwise
it can be explicitly used to switch to this region.
\end{docCommand}

\section{Region \qt{GB}}
\DTLenLocaleHook
\DTLGBLocaleHook

The \file{datatool-GB.ldf} file provides support for region \qt{GB}
(United Kingdom). This defines the \qt{GBP} currency and sets it as
the default currency. The default number group character is a
comma, and the default decimal character is a dot.

If there's no caption hook and you have multiple locales, you will
need to set both the language and region hooks. For example, this
document has:
\begin{dispListing}
\DTLenLocaleHook
\DTLGBLocaleHook
\end{dispListing}

Options may be set with \setlocaleopts{GB}. Available options are
listed below.

\begin{docKey}[GB]{currency-symbol-prefix}{=\meta{boolean}}{initially \opt{false}}
If true, shows the region prefix before the currency symbol if
the \opt{numeric} setting \opt{currency-symbol-style} is not set to \qt{iso}.
The prefix font may be changed to smallcaps or smaller with
\setbaseopt{ numeric = \brackets{ region-currency-prefix = \meta{value} } }
where \meta{value} is \opt{smallcaps} or \opt{smaller}.
\end{docKey}

\begin{dispExample}
\DTLsetup{numeric={region-currency-prefix=smallcaps}}
\DTLsetLocaleOptions{GB}{currency-symbol-prefix}
\DTLdecimaltocurrency{12345.678}{\result}\result
\end{dispExample}


\begin{docKey}[GB]{currency-symbol-position}{=\meta{value}}{initially \opt{before}}
Adjusts the formatting currency style. If \docValue{before}, the
currency symbol is placed before the value. If \docValue{after}, the
currency symbol is placed after the value.
\end{docKey}

\begin{dispExample}
\DTLsetup{numeric={currency-symbol-style=iso}}
\DTLsetLocaleOptions{GB}{currency-symbol-position=after}
\DTLdecimaltocurrency{12345.678}{\result}\result
\end{dispExample}

\begin{docKey}[GB]{currency-symbol-sep}{=\meta{value}}{initially \opt{none}}
Sets the separator to use between the currency symbol (not code) and the value.
Permitted values: \docValue{none} (no space), \docValue{thin-space}
(a thin space), \docValue{space} (a normal space), or
\docValue{nbsp} (a non-breaking space).
\end{docKey}

\begin{dispExample}
\DTLsetLocaleOptions{GB}{currency-symbol-sep=thin-space}
\DTLdecimaltocurrency{12345.678}{\result}\result
\end{dispExample}

\begin{docKey}[GB]{number-style}{=\meta{setting}}{initially \opt{official}}
Sets the number group and decimal characters. The value may be one
of: \docValue{official} (comma number group and decimal point),
\docValue{education} (thin space number group and decimal point), or
\docValue{old} (comma number group and mid dot decimal).
In the case of \docValue{education}, a normal space may also be used
when parsing.
In the case of \docValue{old}, a normal dot may also be used
when parsing.
\end{docKey}

\begin{dispExample}
\DTLsetup{numeric={auto-reformat}}
\DTLsetLocaleOptions{GB}{number-style=education}
\DTLdecimaltocurrency{12345.678}{\result}\result.
(Value: \DTLdatumvalue{\result}.)

\DTLparse{\result}{£12 345.678}\result.
(Value: \DTLdatumvalue{\result}.)

\DTLsetLocaleOptions{GB}{number-style=old}
\DTLdecimaltocurrency{12345.678}{\result}\result.
(Value: \DTLdatumvalue{\result}.)

\DTLparse{\result}{£12,345.678}\result.
(Value: \DTLdatumvalue{\result}.)
\end{dispExample}

The remaining options relate to dates and times, which are still
experimental. You need to enable date and time parsing with
\setbaseopt{ datetime = \brackets{parse} }.

\begin{docKey}[GB]{date-style}{=\meta{style}}{initially \opt{dmyyyy}}
Sets the current date style. The value may be one of:
\docValue{dmyyyy} (day month year), \docValue{mdyyyy} (month day year),
\docValue{yyyymd} (year month day),
\docValue{dmyy} (day month 2-digit year),
\docValue{mdyy} (month day 2-digit year), or
\docValue{yymd} (2-digit year month day).
\end{docKey}

\begin{docKey}[GB]{date-variant}{=\meta{style}}{initially \opt{slash}}
Sets the current numeric date separator.
Allowed values: \docValue{slash} (\code{/}), \docValue{hyphen}
(\code{-}), \docValue{dot} (\code{.}) or
\docValue{dialect} (if the language supports it).
\end{docKey}

\begin{docKey}[GB]{time-variant}{=\meta{style}}{initially \opt{colon}}
Sets the current numeric time separator.
Allowed values: \docValue{colon} (\code{:}), \docValue{dot} (\code{.}) or
\docValue{dialect} (if the language supports it),
\docValue{dialect-colon} (if the language supports it), or
\docValue{dialect-dot} (if the language supports it).
\end{docKey}

\begin{dispExample}
\DTLsetup{datetime={parse}}
\DTLsetLocaleOptions{GB}{
 date-style=dmyyyy,
 date-variant = dialect,
 time-variant = dot
}
\DTLparse\result{Fri 28th Feb 2025 3.45pm GMT}
String: \result.
Data type: \DTLgetDataTypeName{\DTLdatumtype{\result}}.
Numeric value: \DTLdatumvalue{\result}.

\ExplSyntaxOn
\datatool_extract_timestamp:NN \result \l_tmpa_tl
Time-stamp: ~ \l_tmpa_tl
\ExplSyntaxOff
\end{dispExample}

\begin{docCommand}{datatoolGBSetNumberChars}{}
Hook to switch to GB number group and decimal characters.
\end{docCommand}

\begin{docCommand}{datatoolGBcurrencyfmt}{}
Used to format the GBP currency. This supports the currency symbol
prefix.
\end{docCommand}

\begin{docCommand}{DTLGBLocaleHook}{}
Hook to switch to GB settings. This may be added to the captions
hook by \sty{datatool-base}, depending on the settings. Otherwise
it can be explicitly used to switch to this region.
\end{docCommand}

\section{Region \qt{IE}}
\DTLenLocaleHook
\DTLIELocaleHook

The \file{datatool-IE.ldf} file provides support for region \qt{IE}
(Republic of Ireland). This sets \qt{EUR} as the default currency.

If there's no caption hook and you have multiple locales, you will
need to set both the language and region hooks. For example, this
document has:
\begin{dispListing}
\DTLenLocaleHook
\DTLIELocaleHook
\end{dispListing}


Options may be set with \setlocaleopts{IE}. Available options are
listed below.

\begin{docKey}[IE]{currency-symbol-position}{=\meta{value}}{initially \opt{before}}
Adjusts the formatting currency style. If \docValue{before}, the
currency symbol is placed before the value. If \docValue{after}, the
currency symbol is placed after the value.
\end{docKey}

\begin{dispExample}
\DTLsetup{numeric={currency-symbol-style=iso}}
\DTLsetLocaleOptions{IE}{currency-symbol-position=after}
\DTLdecimaltocurrency{12345.678}{\result}\result
\end{dispExample}

\begin{docKey}[IE]{currency-symbol-sep}{=\meta{value}}{initially \opt{none}}
Sets the separator to use between the currency symbol (not code) and the value.
Permitted values: \docValue{none} (no space), \docValue{thin-space}
(a thin space), \docValue{space} (a normal space), or
\docValue{nbsp} (a non-breaking space).
\end{docKey}

\begin{dispExample}
\DTLsetLocaleOptions{IE}{currency-symbol-sep=thin-space}
\DTLdecimaltocurrency{12345.678}{\result}\result
\end{dispExample}

\begin{docKey}[IE]{number-style}{=\meta{setting}}{initially \opt{official}}
Sets the number group and decimal characters. The value may be one
of: \docValue{official} (comma number group and decimal point) or
\docValue{unofficial} (thin space number group and decimal point).
In the case of \docValue{unofficial}, a normal space may also be used
when parsing.
\end{docKey}

\begin{dispExample}
\DTLsetup{numeric={auto-reformat}}
\DTLsetLocaleOptions{IE}{number-style=unofficial}
\DTLdecimaltocurrency{12345.678}{\result}\result.
(Value: \DTLdatumvalue{\result}.)

\DTLparse{\result}{€12 345.678}\result.
(Value: \DTLdatumvalue{\result}.)
\end{dispExample}

The remaining options relate to dates and times, which are still
experimental. You need to enable date and time parsing with
\setbaseopt{ datetime = \brackets{ parse } }.

\begin{docKey}[IE]{date-style}{=\meta{style}}{initially \opt{dmyyyy}}
Sets the current date style. The value may be one of:
\docValue{dmyyyy} (day month year), \docValue{mdyyyy} (month day year),
\docValue{yyyymd} (year month day),
\docValue{dmyy} (day month 2-digit year),
\docValue{mdyy} (month day 2-digit year), or
\docValue{yymd} (2-digit year month day).
\end{docKey}

\begin{docKey}[IE]{date-variant}{=\meta{style}}{initially \opt{slash}}
Sets the current numeric date separator.
Allowed values: \docValue{slash} (\code{/}), \docValue{hyphen}
(\code{-}), \docValue{dot} (\code{.}) or
\docValue{dialect} (if the language supports it).
\end{docKey}

\begin{docKey}[IE]{time-variant}{=\meta{style}}{initially \opt{colon}}
Sets the current numeric time separator.
Allowed values: \docValue{colon} (\code{:}), \docValue{dot} (\code{.}) or
\docValue{dialect} (if the language supports it),
\docValue{dialect-colon} (if the language supports it), or
\docValue{dialect-dot} (if the language supports it).
\end{docKey}

\begin{dispExample}
\DTLsetup{datetime={parse}}
\DTLsetLocaleOptions{IE}{
 date-style=dmyyyy,
 date-variant = dialect,
 time-variant = dot
}
\DTLparse\result{1st June 2025 3.45pm IST}
String: \result.
Data type: \DTLgetDataTypeName{\DTLdatumtype{\result}}.
Numeric value: \DTLdatumvalue{\result}.

\ExplSyntaxOn
\datatool_extract_timestamp:NN \result \l_tmpa_tl
Time-stamp: ~ \l_tmpa_tl
\ExplSyntaxOff
\end{dispExample}

\begin{docCommand}{datatoolIESetNumberChars}{}
Hook to switch to IE number group and decimal characters.
\end{docCommand}

\begin{docCommand}{datatoolIESetCurrency}{}
Hook to switch to the EUR currency, rounding to 2 decimal places,
and redefines \cs{DTLdefaultEURcurrencyfmt} to reflect the 
\opt{currency-symbol-position} setting.
\end{docCommand}

\begin{docCommand}{DTLIELocaleHook}{}
Hook to switch to IE settings. This may be added to the captions
hook by \sty{datatool-base}, depending on the settings. Otherwise
it can be explicitly used to switch to this region.
\end{docCommand}

\section{Region \qt{US}}
\DTLenLocaleHook
\DTLUSLocaleHook

The \file{datatool-US.ldf} file provides support for region \qt{US}
(United States). This defines the \qt{USD} currency and sets it as
the default currency. The default number group character is a
comma, and the default decimal character is a dot.

If there's no caption hook and you have multiple locales, you will
need to set both the language and region hooks. For example, this
document has:
\begin{dispListing}
\DTLenLocaleHook
\DTLUSLocaleHook
\end{dispListing}

Options may be set with \setlocaleopts{US}. Available options are
listed below.

\begin{docKey}[US]{currency-symbol-prefix}{=\meta{boolean}}{initially \opt{false}}
If true, shows the region prefix before the currency symbol if
the \opt{numeric} setting \opt{currency-symbol-style} is not set to \qt{iso}.
The prefix font may be changed to smallcaps or smaller with
\setbaseopt{ numeric = \brackets{ region-currency-prefix = \meta{value} } }
where \meta{value} is \opt{smallcaps} or \opt{smaller}.
\end{docKey}

\begin{dispExample}
\DTLsetup{numeric={region-currency-prefix=smallcaps}}
\DTLsetLocaleOptions{US}{currency-symbol-prefix}
\DTLdecimaltocurrency{12345.678}{\result}\result
\end{dispExample}


\begin{docKey}[US]{currency-symbol-position}{=\meta{value}}{initially \opt{before}}
Adjusts the formatting currency style. If \docValue{before}, the
currency symbol is placed before the value. If \docValue{after}, the
currency symbol is placed after the value.
\end{docKey}

\begin{dispExample}
\DTLsetup{numeric={currency-symbol-style=iso}}
\DTLsetLocaleOptions{US}{currency-symbol-position=after}
\DTLdecimaltocurrency{12345.678}{\result}\result
\end{dispExample}

\begin{docKey}[US]{currency-symbol-sep}{=\meta{value}}{initially \opt{none}}
Sets the separator to use between the currency symbol (not code) and the value.
Permitted values: \docValue{none} (no space), \docValue{thin-space}
(a thin space), \docValue{space} (a normal space), or
\docValue{nbsp} (a non-breaking space).
\end{docKey}

\begin{dispExample}
\DTLsetLocaleOptions{US}{currency-symbol-sep=thin-space}
\DTLdecimaltocurrency{12345.678}{\result}\result
\end{dispExample}

\begin{docKey}[US]{number-style}{=\meta{setting}}{initially \opt{official}}
Sets the number group and decimal characters. The value may be one
of: \docValue{official} (comma number group and decimal point), or
\docValue{unofficial} (thin space number group and decimal point).
In the case of \docValue{unofficial}, a normal space may also be used
when parsing.
\end{docKey}

\begin{dispExample}
\DTLsetup{numeric={auto-reformat}}
\DTLsetLocaleOptions{US}{number-style=unofficial}
\DTLdecimaltocurrency{12345.678}{\result}\result.
(Value: \DTLdatumvalue{\result}.)

\DTLparse{\result}{\$12 345.678}\result.
(Value: \DTLdatumvalue{\result}.)
\end{dispExample}

The remaining options relate to dates and times, which are still
experimental. You need to enable date and time parsing with
\setbaseopt{ datetime = \brackets{parse} }.

\begin{docKey}[US]{date-style}{=\meta{style}}{initially \opt{mdyyyy}}
Sets the current date style. The value may be one of:
\docValue{dmyyyy} (day month year), \docValue{mdyyyy} (month day year),
\docValue{yyyymd} (year month day),
\docValue{dmyy} (day month 2-digit year),
\docValue{mdyy} (month day 2-digit year), or
\docValue{yymd} (2-digit year month day).
\end{docKey}

\begin{docKey}[US]{date-variant}{=\meta{style}}{initially \opt{slash}}
Sets the current numeric date separator.
Allowed values: \docValue{slash} (\code{/}), \docValue{hyphen}
(\code{-}), \docValue{dot} (\code{.}) or
\docValue{dialect} (if the language supports it).
\end{docKey}

\begin{docKey}[US]{time-variant}{=\meta{style}}{initially \opt{colon}}
Sets the current numeric time separator.
Allowed values: \docValue{colon} (\code{:}), \docValue{dot} (\code{.}) or
\docValue{dialect} (if the language supports it),
\docValue{dialect-colon} (if the language supports it), or
\docValue{dialect-dot} (if the language supports it).
\end{docKey}

\begin{dispExample}
\DTLsetup{datetime={parse}}
\DTLsetLocaleOptions{US}{
 date-style=mdyyyy,
 date-variant = dialect,
 time-variant = dot
}
\DTLparse\result{February 28, 2025 3.45pm EST}
String: \result.
Data type: \DTLgetDataTypeName{\DTLdatumtype{\result}}.
Numeric value: \DTLdatumvalue{\result}.

\ExplSyntaxOn
\datatool_extract_timestamp:NN \result \l_tmpa_tl
Time-stamp: ~ \l_tmpa_tl
\ExplSyntaxOff
\end{dispExample}

\begin{docCommand}{datatoolUSSetNumberChars}{}
Hook to switch to US number group and decimal characters.
\end{docCommand}

\begin{docCommand}{datatoolUScurrencyfmt}{}
Used to format the USD currency. This supports the currency symbol
prefix.
\end{docCommand}

\begin{docCommand}{DTLUSLocaleHook}{}
Hook to switch to US settings. This may be added to the captions
hook by \sty{datatool-base}, depending on the settings. Otherwise
it can be explicitly used to switch to this region.
\end{docCommand}

\section{Region \qt{ZA}}
\DTLenLocaleHook
\DTLenZALocaleHook
\DTLZALocaleHook

The \file{datatool-ZA.ldf} file provides support for region \qt{ZA}
(South Africa). This supplies the currency (ZAR) but number formatting
depends on the language, so this requires specific language \&
region files, which should be provided by the applicable language
module. For example, \sty{datatool-english} provides
\file{datatool-en-ZA.ldf}.

If there's no caption hook and you have multiple locales, you will
need to set both the language and region hooks. For example, this
document has:
\begin{dispListing}
\DTLenLocaleHook
\DTLenZALocaleHook
\DTLZALocaleHook
\end{dispListing}


Options may be set with \setlocaleopts{ZA}. Available options are
listed below.

\begin{docKey}[ZA]{currency-symbol-position}{=\meta{value}}{initially \opt{before}}
Adjusts the formatting currency style. If \docValue{before}, the
currency symbol is placed before the value. If \docValue{after}, the
currency symbol is placed after the value.
\end{docKey}

\begin{dispExample}
\DTLsetup{numeric={currency-symbol-style=iso}}
\DTLsetLocaleOptions{ZA}{currency-symbol-position=after}
\DTLdecimaltocurrency{12345.678}{\result}\result
\end{dispExample}

\begin{docKey}[ZA]{currency-symbol-sep}{=\meta{value}}{initially \opt{none}}
Sets the separator to use between the currency symbol (not code) and the value.
Permitted values: \docValue{none} (no space), \docValue{thin-space}
(a thin space), \docValue{space} (a normal space), or
\docValue{nbsp} (a non-breaking space).
\end{docKey}

\begin{dispExample}
\DTLsetLocaleOptions{ZA}{currency-symbol-sep=thin-space}
\DTLdecimaltocurrency{12345.678}{\result}\result
\end{dispExample}

\begin{docKey}[ZA]{number-style}{=\meta{style}}{initially \opt{official}}
Sets the number group and decimal characters. The value may be one
of: \docValue{official} (dot number group and decimal comma) or
\docValue{dialect} (attempt to use the number style for
the module \meta{lang}-ZA, if it's defined, where \meta{lang} is the
current language tag).
\end{docKey}

\begin{dispExample}
\DTLsetLocaleOptions{ZA}{number-style=official}
\DTLdecimaltocurrency{12345.678}{\result}\result
\end{dispExample}
Or if \ldf{en-ZA} has been loaded and the current language is
English:
\begin{dispExample}
\DTLsetLocaleOptions{ZA}{number-style=dialect}
\DTLdecimaltocurrency{12345.678}{\result}\result
\end{dispExample}

The remaining options relate to dates and times, which are still
experimental. You need to enable date and time parsing with
\setbaseopt{ datetime = \brackets{parse} }.

\begin{docKey}[ZA]{date-style}{=\meta{style}}{initially \opt{yyyymd}}
Sets the current date style. The value may be one of:
\docValue{dmyyyy} (day month year), \docValue{mdyyyy} (month day year),
\docValue{yyyymd} (year month day),
\docValue{dmyy} (day month 2-digit year),
\docValue{mdyy} (month day 2-digit year), or
\docValue{yymd} (2-digit year month day).
\end{docKey}

\begin{docKey}[ZA]{date-variant}{=\meta{style}}{initially \opt{hyphen}}
Sets the current numeric date separator.
Allowed values: \docValue{slash} (\code{/}), \docValue{hyphen}
(\code{-}), \docValue{dot} (\code{.}) or
\docValue{dialect} (if the language supports it).
\end{docKey}

\begin{docKey}[ZA]{time-variant}{=\meta{style}}{initially \opt{colon}}
Sets the current numeric time separator.
Allowed values: \docValue{colon} (\code{:}), \docValue{dot} (\code{.}) or
\docValue{dialect} (if the language supports it),
\docValue{dialect-colon} (if the language supports it), or
\docValue{dialect-dot} (if the language supports it).
\end{docKey}

\begin{dispExample}
\DTLsetup{datetime={parse}}
\DTLsetLocaleOptions{ZA}{
 date-style=mdyyyy,
 date-variant = dialect,
 time-variant = dot
}
\DTLparse\result{June 1, 2025 3.45pm SAST}
String: \result.
Data type: \DTLgetDataTypeName{\DTLdatumtype{\result}}.
Numeric value: \DTLdatumvalue{\result}.

\ExplSyntaxOn
\datatool_extract_timestamp:NN \result \l_tmpa_tl
Time-stamp: ~ \l_tmpa_tl
\ExplSyntaxOff
\end{dispExample}
\begin{docCommand}{datatoolZASetNumberChars}{}
Hook to switch to ZA number group and decimal characters.
\end{docCommand}

\begin{docCommand}{datatoolZAcurrencyfmt}{}
Used to format the ZAR currency.
\end{docCommand}

\begin{docCommand}{DTLZALocaleHook}{}
Hook to switch to ZA settings. This may be added to the captions
hook by \sty{datatool-base}, depending on the settings. Otherwise
it can be explicitly used to switch to this region.
\end{docCommand}

\StopEventually{%
  \PrintChanges
  \PrintIndex
}
\end{document}
